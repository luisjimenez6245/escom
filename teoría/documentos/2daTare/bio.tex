\documentclass[12pt]{article}
\usepackage[utf8]{inputenc}
\usepackage{longtable}
\usepackage{multirow}
\usepackage{graphicx}
\graphicspath{ {./images/} }



\renewcommand{\baselinestretch}{1.5}

\title{Turing Awards}
\author{Luis Diego Jiménez Delgado}
\date{Agosto 12 del 2019}

\begin{document}
\maketitle{pskd}

\section{Introduction}
Los premio

\begin{center}
\begin{longtable}{| p{.80\textwidth} | p{.20\textwidth} | p{20cm}} 
\hline
Nombre & Año \\
\hline
\textbf{Geoffrey Hinton} &  \multirow{3}{4em}{\includegraphics{1}} \\ 
2018&   \\ 
&\\
\hline
\textbf{Yoshua Bengio } &  \multirow{3}{4em}{\includegraphics{2}} \\ 
2018&   \\ 
&\\
\hline
\textbf{Yann LeCun } &  \multirow{3}{4em}{\includegraphics{3}} \\ 
2018&   \\ 
&\\
\hline
\textbf{David A. Patterson } &  \multirow{3}{4em}{\includegraphics{4}} \\ 
2017&   \\ 
&\\
\hline
\textbf{John L. Hennessy } &  \multirow{3}{4em}{\includegraphics{5}} \\ 
2017&   \\ 
&\\
\hline
\textbf{Tim Berners-Lee } &  \multirow{3}{4em}{\includegraphics{6}} \\ 
2016&   \\ 
&\\
\hline
\textbf{Whitfield Diffie } &  \multirow{3}{4em}{\includegraphics{7}} \\ 
2015&   \\ 
&\\
\hline
\textbf{Martin Hellman } &  \multirow{3}{4em}{\includegraphics{8}} \\ 
2015&   \\ 
&\\
\hline
\textbf{Michael Stonebraker}  &  \multirow{3}{4em}{\includegraphics{9}} \\ 
2014&   \\ 
&\\
\hline
\textbf{Leslie Lamport } &  \multirow{3}{4em}{\includegraphics{10}} \\ 
2013&   \\ 
&\\
\hline
\textbf{Silvio Micali } &  \multirow{3}{4em}{\includegraphics{11}} \\ 
2012&   \\ 
&\\
\hline
\textbf{Shafrira Goldwasser } &  \multirow{3}{4em}{\includegraphics{12}} \\ 
2012&   \\ 
&\\
\hline
\textbf{Judea Pearl } &  \multirow{3}{4em}{\includegraphics{13}} \\ 
2011&   \\ 
&\\
\hline
\textbf{Leslie Valiant}  &  \multirow{3}{4em}{\includegraphics{14}} \\ 
2010&   \\ 
\hline
\textbf{Charles Thacker}  &  \multirow{3}{4em}{\includegraphics{15}} \\ 
2009&   \\
&\\
\hline
\textbf{Barbara Liskov } &  \multirow{3}{4em}{\includegraphics{16}} \\ 
2008&   \\ 
&\\
\hline
\textbf{Edmund Clarke}  &  \multirow{3}{4em}{\includegraphics{17}} \\ 
2007&   \\ 
&\\
\hline
\textbf{Joseph Sifakis}  &  \multirow{3}{4em}{\includegraphics{18}} \\ 
2007&   \\ 
&\\
\hline
\textbf{E. Allen Emerson}  &  \multirow{3}{4em}{\includegraphics{19}} \\ 
2007&   \\ 
&\\
\hline
\textbf{Frances Elizabeth Allen}  &  \multirow{3}{4em}{\includegraphics{20}} \\ 
2006&   \\
&\\
\hline
\textbf{Peter Naur } &  \multirow{3}{4em}{\includegraphics{21}} \\ 
2005&   \\ 
&\\
\hline
\textbf{Robert Kahn}  &  \multirow{3}{4em}{\includegraphics{22}} \\ 
2004&   \\ 
&\\
\hline
\textbf{Vinton Cerf}  &  \multirow{3}{4em}{\includegraphics{23}} \\ 
2004&   \\ 
&\\
\hline
\textbf{Alan Kay}  &  \multirow{3}{4em}{\includegraphics{24}} \\ 
2003&   \\ 
&\\
\hline
\textbf{Ronald Rivest}  &  \multirow{3}{4em}{\includegraphics{25}} \\ 
2002&   \\ 
&\\
\hline
\textbf{Leonard Adleman}  &  \multirow{3}{4em}{\includegraphics{26}} \\ 
2002&   \\ 
&\\
\hline
\textbf{Adi Shamir}  &  \multirow{3}{4em}{\includegraphics{27}} \\ 
2002&   \\
&\\
\hline
\textbf{Kristen Nygaard } &  \multirow{3}{4em}{\includegraphics{28}} \\ 
2001&   \\ 
&\\
\hline
\textbf{Ole-Johan Dahl}  &  \multirow{3}{4em}{\includegraphics{29}} \\ 
2001&   \\ 
&\\
\hline
\textbf{Andrew Chi-Chih Yao}  &  \multirow{3}{4em}{\includegraphics{30}} \\ 
2000&   \\ 
&\\
\hline
\textbf{Frederick Brooks } &  \multirow{3}{4em}{\includegraphics{31}} \\ 
1999&   \\
&\\
\hline
\textbf{Jim Gray}  &  \multirow{3}{4em}{\includegraphics{32}} \\ 
1998&   \\ 
&\\
\hline
\textbf{Douglas Engelbart}  &  \multirow{3}{4em}{\includegraphics{33}} \\ 
1997&   \\ 
&\\
\hline
\textbf{Amir Pnueli } &  \multirow{3}{4em}{\includegraphics{34}} \\ 
1996&   \\ 
&\\
\hline
\textbf{Manuel Blum } &  \multirow{3}{4em}{\includegraphics{35}} \\ 
1995&   \\ 
&\\
\hline
\textbf{Raj Reddy } &  \multirow{3}{4em}{\includegraphics{36}} \\ 
1994&   \\ 
&\\
\hline
\textbf{Edward Feigenbaum } &  \multirow{3}{4em}{\includegraphics{37}} \\ 
1994&   \\ 
&\\
\hline
\textbf{Juris Hartmanis}  &  \multirow{3}{4em}{\includegraphics{38}} \\ 
1993&   \\ 
&\\
\hline
\textbf{Richard Stearns}  &  \multirow{3}{4em}{\includegraphics{39}} \\ 
1993&   \\ 
&\\
\hline
\textbf{Butler Lampson} &  \multirow{3}{4em}{\includegraphics{39}} \\ 
1992&   \\ 
&\\
\hline
\textbf{Robin Milner}  &  \multirow{3}{4em}{\includegraphics{40}} \\ 
1991&   \\ 
&\\
\hline
\textbf{Fernando Corbató}  &  \multirow{3}{4em}{\includegraphics{41}} \\ 
1990&   \\ 
&\\
\hline
\textbf{William Kahan}  &  \multirow{3}{4em}{\includegraphics{42}} \\ 
1989&   \\ 
&\\
\hline
\textbf{Ivan Sutherland}  &  \multirow{3}{4em}{\includegraphics{43}} \\ 
1988&   \\ 
&\\
\hline
\textbf{John Cocke}  &  \multirow{3}{4em}{\includegraphics{44}} \\ 
1987&   \\ 
&\\
\hline
\textbf{John Hopcroft}  &  \multirow{3}{4em}{\includegraphics{45}} \\ 
1986&   \\ 
&\\
\hline
\textbf{Robert Tarjan}  &  \multirow{3}{4em}{\includegraphics{46}} \\ 
1986&   \\ 
&\\
\hline
\textbf{Richard Karp} &  \multirow{3}{4em}{\includegraphics{47}} \\ 
1985&   \\ 
&\\
\hline
\textbf{Niklaus Wirth} &  \multirow{3}{4em}{\includegraphics{48}} \\ 
1984&   \\ 
&\\
\hline
\textbf{Dennis Ritchie}  &  \multirow{3}{4em}{\includegraphics{49}} \\ 
1983&   \\ 
&\\
\hline
\textbf{Ken Thompson} &  \multirow{3}{4em}{\includegraphics{50}} \\ 
1983&   \\ 
&\\
\hline


\end{longtable}
\end{center}

\textbf{4.Evaluating Costs}
\\An RFP for the procurement of goods or nonprofessional services may request
that a prospective vendor include pricing information in its proposal. \\This pricing
information should be treated as nonbinding, primarily because negotiations may
bring to light additional needs or requirements not identified in the RFP or
because some anticipated services may not be required.
\\Nevertheless, if an RFP for the procurement of goods or nonprofessional
services identifies cost as an evaluation criterion, it must be evaluated.
Following are three different ways in which the cost of a procurement may be
evaluated: \\
This analysis consists of reviewing and evaluating separate
cost elements and proposed profit of: (1) a vendor’s costs or pricing data;
and (2) the judgmental factors applied in projecting from the data to the
estimated costs, in order to form an opinion as to the degree to which the
proposed costs represent what the contract should cost, assuming
reasonable economy and efficiency.
\\This analysis requires that the committee evaluate specific elements of cost, the necessity for certain
costs, the reasonableness of the amounts estimated for the necessary
costs, the reasonableness of allowances for contingencies, the basis used
for allocation of indirect costs, the appropriateness of allocations of
particular indirect costs to the proposed contract, and the reasonableness
of the total cost or price.\\
\textbf{• Price analysis:}
\\ This analysis is made by examining and evaluating a
proposed price without evaluating its separate cost elements and
proposed profit.\\ In making this analysis, consideration must be given to
any differing terms and conditions. Price analysis is used to determine if a
price is reasonable and acceptable and involves an evaluation of the
prices for the same or similar goods or services. \\This analysis requires the
selection committee to evaluate the price submissions of prospective
vendors in the current procurement, prior price quotations and contract
prices charged, prices published in catalogues or price lists, prices
available on the open market, and in-house estimates of cost.\\
\textbf{• Value analysis:} 
\\This analysis evaluates the function of a product and its
related costs in order to determine its inherent worth or value and to
determine if the price is consistent with what the goods or services should
cost. \\This analysis requires that the committee evaluate what the goods or
services provide to the County, the life-cycle costs, whether there are
other ways in which the service or function could be obtained and what it
would cost, and whether there are features of the goods or services that
could be modified or eliminated.
\\The selection committee must determine how cost information will be evaluated
before the evaluation process begins.
\\
\textbf{5.Determining Whether a Proposal is Suitable for Further Consideration}
The selection committee may exclude a vendor from further consideration in the
procurement process if it determines that its proposal is inferior based on scoring
of the evaluation criteria and, therefore, not suitable for further consideration.
\\The committee should not determine which proposals are not suitable on the
basis of a predetermined cut-off score. Rather, those proposals determined to be
excluded from further consideration should be those that are inferior because of
deficiencies that are not easily correctable.
\\A proposal must contain sufficient information so that the selection committee
knows what is being proposed. An informational deficiency should be considered
material, and the proposal inferior, only after the selection committee considers:
\\(1) the detail called for in the RFP; \\(2) whether the omissions make the proposal unsuitable for further consideration or merely inferior; \\(3) the scope and range of
the omissions; \\(4) whether the proposal offers significant cost savings; and \\(5)
the number of off erors in the competitive range. 
Following are examples of deficiencies in a proposal that the selection committee
may determine to be material and to render the proposal not suitable for further
consideration:\\
\textbf{• Fails to satisfy technical objectives:} \\The proposal is incapable of satisfying
the technical objectives of the RFP.\\
\textbf{• Material deficiencies:} \\The proposal contains deficiencies that are so
material as to preclude any possibility of upgrading the proposal except
through major modifications or revisions.\\
\textbf{• Extremely low rating:} \\The proposal is rated so low that any attempt to
upgrade the proposal to an acceptable level would require an
unreasonable and unfair degree of assistance from the County.

\textbf{• Technical issue not clarified:} \\The proposal is technically inferior and the
vendor fails to timely respond to a request for clarification.

\textbf{• Price omissions:} \\The proposal omits a number of required prices.

\textbf{• Feasibility of approach unsupported:} \\The proposal fails to include detailed
information establishing the feasibility of the vendor’s proposed approach.

\textbf{• Addendum not acknowledged:} \\The vendor fails to acknowledge receipt of
an addendum to the RFP where offered prices would have been
significantly increased by the addendum and vendors were advised that
failure to acknowledge the amendment would result in rejection of the
proposal.
\\A vendor should not be excluded if there is a close question as to whether its
proposal is competitive or if information deficiencies in the proposal can be
corrected by information obtained during the negotiation process, particularly if
the deficiencies are the result of deficiencies in the RFP. \\
\textbf{6.Establishing a Short-list of Vendors}
\\Upon completion of the evaluation of the proposals, the selection committee
should assign a score based on evaluation criteria and classify each proposal
either “acceptable for further consideration” or “not suitable for further
consideration.” Thereafter, the committee shall invite two or more vendors it
deems fully qualified, responsible and suitable, to interview with the committee. 
10/18 15-6
\\The County is not required to interview all vendors who submit proposals. The
number of vendors selected depends in part on the size, scope and complexity
of the project, the number of qualified proposals, and the time available to complete the selection process. \\The process becomes cumbersome when more
than five vendors are considered at the interview stage.
If the purchasing agent, on the recommendation of the selection committee,
determines in writing and in his sole discretion that only one vendor is fully
qualified, or that one vendor is clearly more highly qualified and suitable than the
others under consideration, a contract may be negotiated and awarded to that
vendor.
\\Vendors whose proposals are eliminated from further consideration should be
notified by the purchasing agent that their proposals were removed from further
consideration. The notice should be in writing and advise the vendors of the
decision and express appreciation for their participation in the process. 
\end{document}