\documentclass[11pt,a4paper]{report}

\usepackage[utf8]{inputenc}
\usepackage[T1]{fontenc}\usepackage[spanish]{babel}
\usepackage[notes,backend=biber]{biblatex-chicago}

\usepackage{graphicx}
\graphicspath{ {./imagenes/} }

\usepackage{listings}
\usepackage{float}
\usepackage{csquotes}


\lstset{
	  frame=tb,
	  language=Java,
	  aboveskip=3mm,
	  belowskip=3mm,
	  showstringspaces=false,
	  columns=flexible,
	  basicstyle={\small\ttfamily},
	  numbers=none,
	  breaklines=false,
	  breakatwhitespace=true,
	  tabsize=1
}

\begin{document}
	\begin{titlepage}
		\centering
		{\scshape\huge\bfseries INSTITUTO POLITÉCNICO NACIONAL \par}
		\vspace{0.7cm}
		{\scshape\LARGE\bfseries Escuela Superior de Computo \par}
		\vspace{0.5cm}
		{\scshape\Large \par}
		\vspace{1.5cm}
		{\Large\bfseries Reporte: \par}
		{\huge\bfseries Automatas \par}
		\vspace{2cm}
		{\LARGE\itshape Teoría Computacional\par}
		\vspace{0.2cm}
		{\Large Luis Diego Jimenez Delgado\par}
		\vfill
			{\scshape\Large 2CM5 \par}
			\vspace{0.5cm}
			{\scshape\large Profesor: \par}
			{\scshape\Large Genaro Juarez Martinez \par}
		\vfill
		{\large \today}
	\end{titlepage}
 
	\tableofcontents{}

	\chapter{Autómata de cadena con terminación 01}
		\section{Programa}
\begin{lstlisting}
#include <stdio.h>
#include <stdlib.h>
#include "TADPilaDin.h"

/*
Cadena 0000000000011111111111
Compile: gcc app.c TADPilaDin.c -o a.out
*/

FILE *fp;
FILE *fanswer;
FILE *fstates;

void generateString(char *, int);
boolean isValidProcess();
char getChar();
void printStack(stack *from);

int main(int argc, const char **argv)
{
    char *file_name = "./prime.txt";
    if (argc == 2)
    {
        int n = atoi(argv[1]);
        generateString(file_name, n);
    }
    fp = fopen(file_name, "r");
    fstates = fopen("states.txt", "w");
    fanswer = fopen("answer.txt", "w");
    if (isValidProcess())
    {
        printf("Cadena valida\n");
    }
    else
    {
        printf("Cadena no valida\n");
    }
    fclose(fanswer);
    fclose(fstates);
    fclose(fp);
    return 0;
}

boolean isValidProcess()
{
    stack container;
    Initialize(&container);
    char toWork = getChar();
    element e;
    while (toWork != EOF)
    {
        while (toWork == '0')
        {
            e.c = 'x';
            Push(&container, e);
            printStack(&container);
            toWork = getChar();
        }
        while (toWork == '1')
        {
            if (!Empty(&container))
            {
                Pop(&container);
                printStack(&container);
                toWork = getChar();
            }
            else
            {
                Destroy(&container);
                return FALSE;
            }
        }
        toWork = getChar();
    }
    if (Empty(&container))
    {
        Destroy(&container);
        return TRUE;
    }
    else
    {
        Destroy(&container);
        return FALSE;
    }
}

char getChar()
{
    return (char)fgetc(fp);
}

void generateString(char *fileName, int number)
{
    FILE *generate;
    generate = fopen(fileName, "w");
    int i;
    printf("Generado cadena\n");
    if (number % 2 == 1)
        ++number;
    for (i = 0; i < (number / 2); ++i)
    {
        fputc('0', generate);
    }
    if (number % 2 == 1)
        ++number;
    for (i = 0; i < (number / 2); ++i)
    {
        fputc('1', generate);
    }
    fputc('\n', generate);
    fclose(generate);
}

void printStack(stack *from)
{
    stack helper1, helper2;
    Initialize(&helper1);
    Initialize(&helper2);
    printf("\nPila\n");
    while (!Empty(from))
    {
        element el;
        el = Pop(from);
        printf("-------\n");
        printf("---%c---\n", el.c);
        printf("-------\n");
        Push(&helper1, el);
    }
    printf("-------\n\n");
    while (!Empty(&helper1))
    {
        Push(&helper2, Pop(&helper1));
    }
    while (!Empty(&helper2))
    {
        element el;
        el = Pop(&helper2);
        Push(from, el);
    }
    Destroy(&helper1);
    Destroy(&helper2);
}
\end{lstlisting}
			\subsection{Input}
\includegraphics[scale=1.0]{tarea8i}\
			\subsection{Output}
\includegraphics[scale=1.0]{tarea8o}\
\includegraphics[scale=1.0]{tarea8o1}\

	\chapter{Cadena palindroma}
		\section{Programa}
\begin{lstlisting}
#include <stdio.h>
#include <stdlib.h>
#include "TADPilaDin.h"

/*
Cadena palindroma
Compile: gcc app.c TADPilaDin.c -o a.out
*/

char getChar();
void getString();
int getRandomNumber(int number);
int selectType();
int getBool();
void manageProcess();

FILE *fanswer;
FILE *fstates;
int length;

int main(int argc, const char **argv)
{
    int n = 100;
    arc4random();
    if (argc == 2)
    {
        n = atoi(argv[1]);
    }
    length = getRandomNumber(n);
    fanswer = fopen("./answer.txt", "w");
    manageProcess();
    fclose(fanswer);
    return 0;
}
void manageProcess()
{
    stack pila;
    Initialize(&pila);
    int type = selectType();
    element e;
    if (type == 1)
    {
        length = length - 2;
    }
    else
    {
        if (type == 2)
        {
            length = length - 2;
        }
        else
        {
            if (type == 3)
            {
                length = length - 1;
            }
            else
            {
                length = length - 1;
            }
        }
    }
    while (length >= 0)
    {
        if (getBool())
        {
            e.c = '0';
            Push(&pila, e);
            fputc('0', fanswer);
        }
        else
        {
            if (getBool())
            {
                e.c = '1';
                Push(&pila, e);
                fputc('1', fanswer);
            }
        }
        --length;
    }
    if (type == 1)
    {
        e.c = '0';
        Push(&pila, e);
        e.c = 'S';
        Push(&pila, e);
    }
    else
    {
        if (type == 2)
        {
            e.c = '1';
            Push(&pila, e);
            e.c = 'S';
            Push(&pila, e);
        }
        else
        {
            if (type == 3)
            {
                e.c = '0';
                Push(&pila, e);
            }
            else
            {
                e.c = '0';
                Push(&pila, e);
            }
        }
    }
    while (!Empty(&pila))
    {
        e = Pop(&pila);
        fputc(e.c, fanswer);
    }
    Destroy(&pila);
}
int selectType()
{
    return getRandomNumber(5);
}

int getBool()
{
    return getRandomNumber(2);
}

int getRandomNumber(int number)
{
    int res = rand();
    if (number > 0)
        res = rand() % number;
    return res;
}
\end{lstlisting}
			\subsection{Input}
\includegraphics[scale=1.0]{tarea9i}\
			\subsection{Output}
\includegraphics[scale=.80]{tarea9o}\
\includegraphics[scale=.60]{tarea9o1}\
	\chapter{S->iCtSA \ A-> eS}
		\section{Programa}
\begin{lstlisting}
#include <stdio.h>
#include <stdlib.h>
/*
if(c){s}s
Compile: gcc app.c
*/
int getBool();
void generateString();
void getString();
int getRandomNumber(int number);

FILE *fanswer;
FILE *fstates;
int length;


int main(int argc, const char **argv)
{
    int n = 100;
    arc4random();
    if (argc == 2)
    {
        n = atoi(argv[1]);
    }
    length = getRandomNumber(n);
    fanswer = fopen("./answer.c", "w");
    fstates = fopen("./states.txt", "w");

    generateString();
    fclose(fanswer);
    fclose(fstates);
}

void generateString()
{
    printf("length: %i\n", length);
    while (length >= 0)
    {
       getString();
    }
    fputc('\n', fanswer);
}

void getString(){
    if (getBool())
    {
        --length;
        fputs("if(/*condition*/)\n{\n", fanswer);
        fputs("S->iCtSA\n", fstates);
        if(getBool())
        {
            getString();
        }
        fputs("\n}\n", fanswer);
        if(getBool())
        {
            fputs("A-> eS\n", fstates);   
            fputs("else\n{\n", fanswer);
            if(getBool()){
                getString();
            }
            fputs("\n}\n", fanswer);
        }
        else{
            fputs("A-> \n", fstates);   
        }
    }
}
int getBool()
{
    return getRandomNumber(2);
}

int getRandomNumber(int number)
{
    int res = rand();
    if (number > 0)
        res = rand() % number;
    return res;
}
\end{lstlisting}
			\subsection{Input}
\includegraphics[scale=.60]{tarea10i}\
			\subsection{Output}
\includegraphics[scale=.60]{tarea9o}\
\includegraphics[scale=.60]{tarea9o1}\
	\chapter{Maquina de Turing con Funcionalidad  }
		\section{Programa}
		    \begin{lstlisting}
#include <stdio.h>
#include <stdlib.h>
#include "TADListaDL.h"

/*
Maquina de Turing para 000111
Compile: gcc app.c TADListaDL.c -o a.out
*/

typedef struct container
{
    int to;
    char change;
    int state;
} container;

FILE *fp;
FILE *fanswer;
FILE *fstates;
int aState;

container evaluation(char c, int state);
container evaluation0(int state);
container evaluation1(int state);
container evaluationX(int state);
container evaluationY(int state);
container evaluationB(int state);

boolean validChar(char c);
boolean isValidProcess();

char getChar();

void stateChange(int from, int to);
void generateString();
void generateString(char *, int);
void printContainer(container c);
void printChanges(lista* l);

int getBool();
int getRandomNumber(int number);

int main(int argc, const char **argv)
{
    char *file_name = "./prime.txt";
    if (argc == 2)
    {
        int n = atoi(argv[1]);
        generateString(file_name, n);
    }
    fp = fopen(file_name, "r");
    fstates = fopen("states.txt", "w");
    fanswer = fopen("answer.txt", "w");
    if (isValidProcess())
    {
        printf("Cadena valida\n");
    }
    else
    {
        printf("Cadena no valida\n");
    }
    fclose(fanswer);
    fclose(fstates);
    fclose(fp);
    return 0;
}

boolean isValidProcess()
{
    lista l;
    Initialize(&l);
    container con;
    elemento e;
    posicion p;
    int counter = 0;
    con.state = 0;
    char c = getChar();
    e.c = c;
    Add(&l, e);
    con.to = 1;
    while (con.state != 4 && validChar(c))
    {
        printChanges(&l);
        aState = con.state;
        con = evaluation(c, con.state);
        if (con.state >= 4)
        {
            stateChange(aState, 4);
            break;
        }
        stateChange(aState, con.state);
        e.c = con.change;
        Replace(&l, ElementPosition(&l, counter + 1), e);
        counter += con.to;
        if (con.to == 1)
        {
            if ((Size(&l) - 1) < counter)
            {
                c = getChar();
                if (validChar(c))
                {
                    e.c = c;
                    Add(&l, e);
                }
                else
                {
                    c = 'B';
                }
            }
            else
            {
                p = ElementPosition(&l, counter + 1);
                c = Position(&l, p).c;
            }
        }
        else
        {
            if (Size(&l) < counter)
            {
                c = 'B';
            }
            else
            {
                p = ElementPosition(&l, counter + 1);
                c = Position(&l, p).c;
            }
        }
    }
    if (con.state == 4 && !validChar(getChar()))
    {
        while (!Empty(&l))
        {
            p = First(&l);
            fputc(Position(&l, p).c, fanswer);
            Remove(&l, p);
        }
        Destroy(&l);
        return TRUE;
    }
    Destroy(&l);
    return FALSE;
}

container evaluation(char c, int state)
{
    container res;
    if (c == '0')
    {
        res = evaluation0(state);
    }
    else if (c == '1')
    {
        res = evaluation1(state);
    }
    else if (c == 'X')
    {
        res = evaluationX(state);
    }
    else if (c == 'Y')
    {
        res = evaluationY(state);
    }
    else if (c == 'B')
    {
        res = evaluationB(state);
    }
    return res;
}

container evaluation0(int state)
{
    container res;
    if (state == 0)
    {
        res.change = 'X';
        res.state = 1;
        res.to = 1;
    }
    else if (state == 1)
    {
        res.change = '0';
        res.state = 1;
        res.to = 1;
    }
    else if (state == 2)
    {
        res.change = '0';
        res.state = 2;
        res.to = -1;
    }
    return res;
}

container evaluation1(int state)
{
    container res;
    res.state = 5;
    if (state == 1)
    {
        res.change = 'Y';
        res.state = 2;
        res.to = -1;
    }
    return res;
}

container evaluationX(int state)
{
    container res;
    res.state = 5;
    if (state == 2)
    {
        res.change = 'X';
        res.state = 0;
        res.to = 1;
    }
    return res;
}

container evaluationY(int state)
{
    container res;
    res.change = 'Y';
    res.to = 1;
    if (state == 0)
    {
        res.state = 3;
    }
    else if (state == 1)
    {
        res.state = 1;
    }
    else if (state == 2)
    {
        res.state = 2;
        res.to = -1;
    }
    else if (state == 3)
    {
        res.state = 3;
    }
    return res;
}

container evaluationB(int state)
{
    container res;
    res.state = 5;
    if (state == 3)
    {
        res.change = 'B';
        res.state = 4;
        res.to = 1;
    }
    return res;
}

char getChar()
{
    return (char)fgetc(fp);
}

void generateString(char *fileName, int number)
{
    FILE *generate;
    generate = fopen(fileName, "w");
    int i;
    printf("Generado cadena\n");
    if (number % 2 == 1)
        ++number;
    for (i = 0; i < (number / 2); ++i)
    {
        fputc('0', generate);
    }
    if (number % 2 == 1)
        ++number;
    for (i = 0; i < (number / 2); ++i)
    {
        fputc('1', generate);
    }
    fputc('\n', generate);
    fclose(generate);
}

void printContainer(container c)
{
    printf("\nc: %c\n", c.change);
    printf("state: %d \n", c.state);
    printf("to: %i\n", c.to);
}

boolean validChar(char c)
{
    return (c != '\n' && c != '\t' && c != EOF && c != ' ');
}

void printChanges(lista *l)
{
    int i = 0;
    printf("Cinta \n");
    for(i  = 0; i < Size(l); ++i)
    {
        printf("-------\n");
        printf("---%c---\n", (Position(l,ElementPosition(l, i + 1)).c));
        printf("-------\n");
    }
    printf("-------\n\n");

}

void stateChange(int from, int to)
{
    fprintf(fstates, "q%i -> q%i\n", from, to);
}

int getBool()
{
    return getRandomNumber(2);
}

int getRandomNumber(int number)
{
    int res = rand();
    if (number > 0)
        res = rand() % number;
    return res;
}

	        \end{lstlisting}

			\subsection{Input}
				\includegraphics[scale=1.0]{turginI1}\
			\subsection{Output}
				\includegraphics[scale=1.0]{turingO1}\
				\includegraphics[scale=1.0]{turingO2}
\end{document}