\documentclass[12pt]{article}
\usepackage[utf8]{inputenc}
\usepackage{longtable}
\usepackage{multirow}
\usepackage{graphicx}
\graphicspath{ {./author/} }


\renewcommand{\baselinestretch}{1.5}

\title{Adolfo Guzmán Arenas(1943- )}
\author{Luis Diego Jiménez Delgado}
\date{Agosto 13 del 2019}

\begin{document}

\maketitle

\begin{center}
    
\includegraphics{a}
\end{center}
Ingeniero en Comunicaciones y Electrónica de la Escuela Superior de Ingeniería y Mecánica y Eléctrica del Instituto Politécnico Nacional (IPN).
 Obtuvo su Maestría y su Doctorado en Ciencias de la Computación en el Instituto Tecnológico de Massachusetts (MIT), en Cambridge, Massachusetts, EE.UU. 
 Fue profesor del Departamento de Ingeniería Eléctrica del MIT; del Departamento de Inteligencia Mecánica de la Universidad de Edimburgo; del Centro de Investigación y Estudios Avanzados del IPN, donde fundó la Maestría 
 y Doctorado en Computación; del Instituto de Investigación en Matemáticas Aplicadas y Sistemas, de la UNAM, donde fue Jefe del Departamento de Computación; y de la Unidad Interdisciplinaria (UPIICSA) del IPN. Fue Director del Centro 
 Científico IBM para América Latina, IBM de México, S.A. Ha sido 
 Investigador Senior de la empresa MicroElectronics and Computer Corporation (MCC); Vicepresidente de Ingeniería en International Software Systems, y 
 fundador y Presidente de SoftwarePro International, empresa en Austin, Texas, dedicada al desarrollo de paquetes comerciales y herramientas de Ingeniería de Software. En 1994 recibió el Premio Nacional de 
 Informática, que otorga la Academia Mexicana de Informática. Recibió en 1996 de manos del Presidente Zedillo el Premio Nacional de Ciencias y Artes. Y de sus mismas manos, en 1997, la Presea “Lázaro Cárdenas”.
Fundó en 1996 el Centro de Investigación en Computación (CIC) del IPN y lo dirigió hasta 2002. Adolfo es miembro de la Academia de Ingeniería, la Academia Mexicana de Ciencias y el Consejo Consultivo de Ciencias. Es Doctor Honoris Causa del Instituto Nacional de Astrofísica, Óptica y Electrónica. Es Fellow of the Association for Computing Machinery (ACM), y Fellow of the Institute of Electrical and Electronic Engineers (IEEE). En el CIC trabaja en el uso de Inteligencia Artificial en análisis de textos (y representación del conocimiento), procesamiento semántico y aplicaciones de sistemas de información
\\
\textbf{Refencias:}
[1]Genaro J. Mart ́ınez1,2, Juan C. Seck-Tuoh-Mora3 Sergio V. Chapa-Vergara4, Christian Lemaitre 5(2019) Brief Notes and History Computing in Mexico during 50 years∗, https://arxiv.org/pdf/1905.07527.pdf

\end{document}