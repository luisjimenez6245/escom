\documentclass[12pt]{article}
\usepackage[utf8]{inputenc}
\usepackage{longtable}
\usepackage{multirow}


\renewcommand{\baselinestretch}{1.5}

\title{Harold V. McIntosh (1929-2015)}
\author{Luis Diego Jiménez Delgado}
\date{Agosto 13 del 2019}

\begin{document}

\maketitle

\textbf{México de 1900}
\\No se puede hablar de computación del país sin hablar de Harold V. McIntosh. Se considera como el pionero de la computación del país, por ejemplo se hace regencia a muchas de las investigaciones que este enigmático personaje realizó hacer de todo lo referente a la computación en el país, realizo investigaciones en estructuras de datos, sistemas operativos.
Las investigaciones de Macintosh llegaron a muchos lugares teniendo influencias grandes en algunas instituciones del país como PEMEX. 
\\
\textbf{La computación en México }
\\Se considera que México comenzó su investigación en la computación desde cero, mientras que en el país vecino la computación crecía a pasos exponenciales ya que el internet estaba empezando, la guerra fría seguía.  La primera computadora en el país fue instalada en lo que hoy en día es Ciudad Universitaria de Universidad Nacional Autónoma de México en el año de 1958, era una IBM que fue traída por un gran influencia de McIntosh considerando que después las damas grandes universidades del país iban a ir obteniendo su equipo de cómputo para ir formando sus propias redes de computadoras. 
Investigaciones como PLOT, que era una gráficadora,  fue uno de los grandes proyectos de McIntosh que influyó en toda America latina. Mientras que el internet iba creciendo la computación en México se fue estacando, por así decirlo los problemas socioeconómicos del país no han permitido que la computación se desarrolle del todo en el país. El país se está quedando estancando por la falta de investigaciones, de la creación de ciencia, de la falta de apoyo a los investigadores, por la falta de apoyo a la computación en el país, y no solo a esa rama tan joven de la ciencia, si no que a todas las que se derivan de esta. Es importante recalcar que se requiere cambiar muchas cosas para que las investigaciones tan importantes en computación como las de el doctor McIntosh.
\\
\textbf{Refencias:}
[1]Genaro J. Mart ́ınez1,2, Juan C. Seck-Tuoh-Mora3 Sergio V. Chapa-Vergara4, Christian Lemaitre 5(2019) Brief Notes and History Computing in Mexico during 50 years∗, https://arxiv.org/pdf/1905.07527.pdf
\end{document}