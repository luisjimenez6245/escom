\documentclass[a4paper]{article}
\usepackage[spanish]{babel}
\usepackage[utf8]{inputenc}


\title{Practica 1 \newline \newline
	  \large \textbf{ Teoría computacional N, Profesor: Genaro Martinez} }

\author{\Large \textbf{Alumno:} \\ Luis Diego Jiménez Delgado \\ 
		\texttt{2CM5, luijsimenez6245@hotmail.com}}

\begin{document}
	\begin{titlepage}
		\maketitle
	\end{titlepage}
	{ \large \tableofcontents }
	\newpage
        \section{Cadenas Binarias}
        Tendencia de los números primos
Objetivo del programa: Encontrar un patrón cuando se generan números primos de entre en el intervalo (1,1000 000)
Proceso: 
\begin{itemize}
    \item Recibe un numero n <=1 000 000 y n>1
    \item Guarda los números primos desde n=1 hasta n
    \item Convierte los números a binario
    \item Se lee la cantidad de 1’s aparecen por número y se guarda esa cantidad por posición
    \item  Se representan estos resultados en una gráfica de posición contra cantidad de unos por número primo.
    \end{itemize}
Se escogen los números que sean primos en un rango en el intervalo [2,n] donde n es un numero proporcionado por el usuario. Se aceptan los números dados sí y solo sí n>1, después se verifica que desde 2 hasta n no existan números que dividan a n, es decir, el número módulo i (número dentro del rango), tiene que ser distinto de cero excepto cuanto i es igual al número.
Segunda etapa: Los números se convierten a binario usando recursividad cuyo caso base es cuando el valor del número es menor a 2 y en el caso recursivo se envía el número divido entre dos.
Tercera etapa: Para cada número generado se realiza un conteo de las ocurrencias del número 1 y se les asigna una posición por número. Estos datos se salvan en un archivo y posteriormente se grafican.
Se obtiene una conclusión: este método no puede generalizarse para encontarr tendencia en los números primos, debido a que no parece tener un patro en el intervalo 1 - 1,000,000.
            \subsection{Espaciadas}
            
			\subsection{Juntas}
			\subsection{Numeros Primos}
		\section{Automata de cadenas}
			En este programa se verifican a través de un autómata las cadenas de que simulan ser transacciones y consiste en lo siguiente: 
			\begin{itemize}
\item El autómata comienza si tiene un número aleatorio que 
\item Se generan los datos de entrada.
\item Espera de dos segundos.
\item Se hace la lectura de las cadenas generadas.
\item En el bloque ACK se generan dos archivos 
\item  Se genera un número aleatorio entre 0 y 1 (inclusive), y se regresa al paso 1. 
\end{itemize}
Primeramente, se genera un número aleatorio entre 0 y 1 que nos indicará si el autómata comienza o no.
Posteriormente, se crea (aleatoriamente) un archivo como se muestra en la figura 2 ``datain.txt''  que contenga 1 cadena con 100000 caracteres cada una.
Después, se hace una lectura cadena a cadena de este archivo. Por cada una de las cadenas se valida que la cantidad de 1 y ceros tengan paridad. 
En caso de que tengan paridad se crea un archivo ``accepted.txt'' (véase la figura 3), el cual contiene la cantidad de cadenas aceptadas y el número de proceso en el que se encontraron. 
En el otro caso, se crea un archivo ``rejected.txt'' (véase la figura 4), el cual contiene la cantidad de cadenas rechazadas y el número de proceso en el que se encontraron. 
Luego, se tiene un archivo ``history.txt'' (véase la figura 5 y 6), el cual nos indica el estado del autómata por proceso, y finalmente se genera otro número aleatorio que nos indicará si nuevos datos van a entrar o el programa finaliza. 


\end{document}  